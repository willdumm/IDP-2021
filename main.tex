\documentclass{paper}

%%%% Patch to make lineno work nicely with amsmath
% https://tex.stackexchange.com/a/461192
% \usepackage{lineno}
\usepackage{amsmath}  %% <- after lineno
\usepackage{etoolbox} %% <- for \cspreto, \csappto
% %% Patch 'normal' math environments:
% \newcommand*\linenomathpatch[1]{%
%   \cspreto{#1}{\linenomath}%
%   \cspreto{#1*}{\linenomath}%
%   \csappto{end#1}{\endlinenomath}%
%   \csappto{end#1*}{\endlinenomath}%
% }
% \linenomathpatch{equation}
% \linenomathpatch{gather}
% \linenomathpatch{multline}
% \linenomathpatch{align}
% \linenomathpatch{alignat}
% \linenomathpatch{flalign}
% \linenumbers%
% %%%% end patching

\usepackage{amsfonts}
\usepackage{amssymb}
\usepackage{amsthm}
\usepackage{graphicx}
\usepackage[hidelinks]{hyperref}
\usepackage[inline]{showlabels}
\renewcommand{\showlabelfont}{\tiny\sffamily}

\newtheorem{lemma}{Lemma}
\newtheorem{prop}{Proposition}
\newtheorem{thm}{Theorem}
\newtheorem{prob}{Problem}
\newtheorem{defn}{Definition}
\newtheorem{obs}{Observation}
\newtheorem{alg}{Algorithm}

\newcommand{\median}{\operatorname{median}}



% http://bytesizebio.net/2013/03/11/adding-supplementary-tables-and-figures-in-latex/
\newcommand{\beginsupplement}{%
        \setcounter{table}{0}
        \renewcommand{\thetable}{S\arabic{table}}%
        \setcounter{figure}{0}
        \renewcommand{\thefigure}{S\arabic{figure}}%
     }

\hyphenation{Ge-nome Ge-nomes hyper-mut-ation through-put}
% disable bibliography
\renewcommand{\cite}[2][]{}
\renewcommand{\bibliography}[1]{}
\renewcommand{\bibliographystyle}[1][]{}
\title{Individual Development Plan 2021}
\author{Will Dumm}

\begin{document}
\maketitle



\section*{Career Goals}

\subsubsection*{Long-Term and 10-year Goals}
In the long term, my goal is to be productive in a flexible, self-directed but collaborative, location-agnostic position contributing to an interesting project in science or applied math.
Getting there will be mostly a journey of self-cultivation, rather than credentialing and experience building.
My current position offers all the freedom that I desire, but I must nurture my own creativity both inside and outside of work to take full advantage of that flexibility.
I think this long-term goal is achievable within the next ten years.

\subsubsection*{This Year's Goals}
Within the next year, my goal is to identify and execute a set of milestones for promotion in my position. These milestones will include:
\begin{itemize}
    \item Becoming a proficient c++ programmer
    \item Building familiarity with relevant data formats and structures, such as the intricacies of variant calling format and the Usher Mutation Annotated Tree
    \item Completion, at least in the scope of the history-DAG and parsimony, of the Parsimony Plateau paper as a coherent, well-rounded, and useful document
    \item Implementation of useful applications of the history-DAG, including
        \begin{itemize}
            \item a deep parsimony search using Usher utilities, applied to SARS-CoV-2 data.
                It will be interesting to focus on how much better we can do than the `accepted' ancestral tree for SARS-CoV-2 constructed by Usher.
            \item a useful application to GCTree inference, hopefully optimizing inference via the DAG structure.
        \end{itemize}
    \item Progressing in understanding and implementation of the ideas described in the `deeply sampled' grant application.
\end{itemize}

\section*{Development of Project-Specific Knowledge}
\subsubsection*{Project Description}
My projects for the foreseeable future revolve around the description of a compact object (the history-DAG) capable of describing an ensemble of labeled trees, organized by associations of internal labels with leaf label paritions.
In the near-term, this will involve exploring the properties of the history-DAG constructed on weighted trees, especially maximum parsimony trees.
In the medium-term, I will focus on developing useful applications for this object.
For example, properties of the history-DAG could be useful in expediting a search for maximally parsimonious trees.
The history-DAG will likely also be useful in compactly expressing a large number of maximally parsimonious trees, and perhaps as a way of characterizing the space of all max-parsimony trees on fixed taxa.
I will also explore ways to use the history-DAG to improve inference in the package GCTree.
Immediate benefits may be realized by using the history-DAG to efficiently utilize a larger number of maximally parsimonious trees in the existing inference pipeline.
It may also be possible to streamline GCTree inference algorithms by making use of the structure of the history-DAG.

As a related project, I will seek to extend the group's work on the subsplit-DAG to allow use of the history-DAG or another new structure to express the search space in the variational inference problem described in the  `densely sampled' grant application.

\subsubsection*{New Skills and Knowledge Gained}
\begin{itemize}
    \item Biology background, including basic immunology and phylogenetics
    \item Mathematical phylogenetic background, including parsimony, basic understanding of applications of MCMC and variational inference to Bayesian phylogenetics
    \item Re-familiarization with Python, DAG-related idioms, shell, and cluster use
\end{itemize}
\subsubsection*{Skills and Knowledge Needed}
The following skills will be needed to tackle project objectives:
\begin{itemize}
    \item Familiarization with c++, including basic knowledge, productive problem-specific idioms, and toolchain proficiency
    \item More familiarity with variational inference applied to phylogenetics
\end{itemize}

\section*{Career Skills}

\subsubsection*{Communication Skills}
So far, opportunities to practice and improve communications skills have been limited to writing and informal/impromptu verbal communication of project ideas and questions.
I could improve my general readiness to explain project ideas without preparation.
One way I may do this is by verbalizing each day a new idea that I've had during that day, even if just to myself.
This will also help me prepare for more formal exercises in verbal communication, such as group meeting presentations.

The most immediate and accessible improvements in writing skills will be automatic, as I become more familiar with the notation and vocabulary that is most common in phylogenetics, and that works best with my project.
I also intend to mindfully implement Erick's suggestions for writing from the group wiki.

\subsubsection*{Other Opportunities for Improvement}
I expect to improve my time management skills:
\begin{itemize}
    \item Keeping sight of immediate goals for each day or subproject, and staying on track
    \item Knowing better when to shift attention when progress on a task slows
    \item Having a more complete array of useful tasks in mind, and better recognizing directions with high potential for progress
\end{itemize}
I can address the first two items by pausing to check in with my goals and progress throughout the day.
The third can be helped by continuing to write down any ideas or questions that come up while working on other tasks.

\subsubsection*{Opportunities for Contacts and Collaboration}
I am looking forward to communicating with authors of Usher about my project, when I have ideas that are ready to share, and questions that require their help.
Erick also mentioned communicating with Mike Steel regarding parsimony on the history-DAG.
% \begin{figure}[h]
% \centering
% \includegraphics[width=0.35\textwidth]{figures/subsplit.pdf}
% \caption{\
% A subsplit structure.
% }%
% \label{fig:subsplit}
% \end{figure}

\nocite{*}
\bibliographystyle{plain}
\bibliography{main}


% \clearpage
% \section*{Supplementary Materials}
% \beginsupplement
% Supplementary text and figures here.


\end{document}
